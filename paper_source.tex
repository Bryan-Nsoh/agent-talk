% Auto-generated concatenation of the paper's TeX sources
% Generated: 2025-10-10T13:11:58Z
% Repo: agent-talk
% Build figures before compiling this standalone source:
%   ./scripts/build_figures.sh
%   ./scripts/build_diagrams.sh
% Then compile the paper inside cert-talk-paper as usual:
%   cd cert-talk-paper && latexmk -pdf -silent main.tex

\documentclass{article}
\usepackage{arxiv}

\usepackage[T1]{fontenc}
\usepackage[utf8]{inputenc}
\usepackage{lmodern}
\usepackage{microtype}
\usepackage{amsmath,amssymb}
\usepackage{graphicx}
\usepackage{booktabs}
\usepackage{caption}
\usepackage{subcaption}
\usepackage{natbib}
\usepackage[dvipsnames]{xcolor}
\usepackage{listings}
\usepackage{tikz}
\usetikzlibrary{arrows.meta,positioning,calc,fit}
\usepackage{url}
\usepackage[hidelinks]{hyperref}
\usepackage{float} % [H] exact placement when truly needed
\usepackage{placeins} % \FloatBarrier to keep floats near text
\usepackage{needspace} % avoid awkward whitespace before figures
% Tighter float spacing for in-text clarity
\setlength{\textfloatsep}{8pt plus 2pt minus 2pt}
\setlength{\floatsep}{8pt plus 2pt minus 2pt}

\setcitestyle{numbers,sort&compress}

\lstdefinelanguage{json}{
  showstringspaces=false,
  breaklines=true,
  columns=fullflexible
}

\lstset{
  language=json,
  basicstyle=\ttfamily\footnotesize,
  numbers=left,
  numbersep=6pt,
  frame=single,
  rulecolor=\color{black},
  breaklines=true,
  showstringspaces=false,
  columns=fullflexible,
  backgroundcolor=\color{black!1},
  xleftmargin=1.2em,
  xrightmargin=1.2em
}

\newcommand{\intuition}[1]{\medskip\noindent\textbf{Intuition.} #1\medskip}

\title{Certificate Talk: Minimal, Proof-Carrying Communication for Two Agents With Private Maps}

\author{Beta Agent Talk Team}

\date{October 9, 2025}

\renewcommand{\headeright}{Draft}
\renewcommand{\undertitle}{Method Note}
\renewcommand{\shorttitle}{Certificate Talk}

\begin{document}
\begin{document}
\maketitle
\maketitle

\begin{abstract}
We study two agents with private obstacle maps that must decide whether a safe path exists between a common start and goal. CertTalk is a deterministic dialogue where every exchange ends in a path or cut certificate that both a human and an oracle can replay. Witness bits and fast certificate replies cut typical transcripts to three to five messages while keeping every packet well under 256~bytes. This note focuses on the protocol and implementation details that make those certificates possible.
\end{abstract}

\keywords{certificate protocols \and multi-agent systems \and verifiable communication}

% ===== BEGIN: sections/intro.tex =====
\section{Why certificates instead of full maps?}
\intuition{Two delivery robots share a radio channel that is barely wide enough for proof objects. CertTalk shows that small, typed exchanges can replace the old plan of streaming raw maps.}

Imagine two couriers who must deliver a heavy meal together. Robot~A carries thermal sensors that flag hot hazards such as steam vents and transformer leaks, while Robot~B scans the ground for sinkholes and torn pavement. The city grid is only $10\times 10$ cells in our test bed, yet broadcasting full sensor imagery still costs several kilobytes and multiple seconds per attempt. More importantly, an engineer reading the resulting logs would still not know which obstacles caused the decision to abort or proceed.

CertTalk offers a different contract. Each message has a rigid JSON skeleton. Proposals are literal artifacts: either a compressed path from the restaurant to the destination, or a compressed cut that severs the two locations. The responder either returns explicit conflicts or upgrades the proposal to a certificate that the initiator acknowledges. A short transcript suffices to convince a third-party oracle that the decision was correct. Because every field is typed and checksummed, the logs remain compact while revealing the entire reasoning chain.

This document walks through the protocol in detail. We begin with a running example, introduce the vocabulary used by the encoders, describe the system model and message flow, and finally document the finite-state agents and the oracle checker. Later versions will reincorporate empirical results and related work once the method section is fully settled.
% ===== END: sections/intro.tex =====
% ===== BEGIN: sections/artifacts_overview.tex =====
\section{Certificates at a glance}
\intuition{Before diving into the protocol, meet the two proof objects that a conversation produces: either a path certificate that proves reachability or a cut certificate that proves separation.}

\subsection*{PATH\_CERT (seed 247)}
\begin{lstlisting}
{"t":"PATH_CERT",
 "p":{"runs":"QoVBgkKCRA==","digest16":22270,"signed_by":["B"]},
 "s":"v1","q":0,"c":49459}
\end{lstlisting}
\begin{itemize}
  \item \textbf{t} identifies the packet as a path certificate.
  \item \textbf{runs} holds the RLE4-compressed path (Right$\times$3, Down$\times$4, Right$\times$2, Down$\times$1).
  \item \textbf{digest16} is the CCITT checksum of the raw run bytes; it detects corruption.
  \item \textbf{signed\_by} lists the issuer. Here Robot~B validated the proposal and issued the certificate.
  \item \textbf{s}, \textbf{q}, \textbf{c} form the envelope: schema version, sender sequence number, and CRC16 of the canonical JSON.
\end{itemize}

\subsection*{CUT\_CERT (seed 123)}
\begin{lstlisting}
{"t":"CUT_CERT",
 "p":{"cells":"AAACAAEAAf4AAQ==","k":4,"digest16":46585,"witness":"AA==","signed_by":["B"]},
 "s":"v1","q":0,"c":4584}
\end{lstlisting}
\begin{itemize}
  \item \textbf{cells} is the DELTA16 encoding of the four separator cells that block all routes.
  \item \textbf{k} records the cell count, while \textbf{digest16} checks the raw bytes.
  \item \textbf{witness} packs one bit per cell (0=A, 1=B) so the responder knows which robot attests each obstacle.
  \item \textbf{signed\_by} again records the issuer. Robot~B could issue the certificate immediately because the separator was locally valid.
  \item The envelope mirrors the path case: schema, monotone sequence, and CRC16 guard the outer message.
\end{itemize}

A full transcript is simply the lead-up to one of these artifacts. The next sections introduce the world in which the robots operate and the conversation they follow to reach these proofs.
% ===== END: sections/artifacts_overview.tex =====
% ===== BEGIN: sections/physical_world.tex =====
\section{The physical world}
\intuition{Both agents live on the same $10\times 10$ Manhattan grid. Each sees a different slice of the obstacles and neither ever observes the union directly.}

\subsection*{Coordinate frame}
Cells are indexed by $(x,y)$ with $x$ increasing eastward and $y$ increasing southward. The restaurant $R$ is fixed at $(0,0)$ and the destination $F$ sits at $(9,9)$. Movements are 4-neighbour only.

\subsection*{Private sensor views}
\begin{minipage}{0.32\linewidth}
\centering
\begin{verbatim}
Robot A
R . . . . . . . . .
. H . . . . . . . .
. . . H . . . . . .
. . . H . . . . . .
. . . H . . . . . .
...
\end{verbatim}
\end{minipage}
\begin{minipage}{0.32\linewidth}
\centering
\begin{verbatim}
Robot B
R . . . . . . . . .
. . . . . . S . . .
. . . . . . S . . .
. . . . . . S . . .
. . . . . . S . . .
...
\end{verbatim}
\end{minipage}
\begin{minipage}{0.32\linewidth}
\centering
\begin{verbatim}
Union U
R . . . . . . . . .
. H . . . . S . . .
. . . H . . S . . .
. . . H . . S . . .
. . . H . . S . . .
...
\end{verbatim}
\end{minipage}

Robot~A’s thermal camera marks hot hazards (H) along column three; Robot~B’s ground radar discovers structural hazards (S) along column six. Only the oracle (and later the certificate) reasons over the union $U = A \cup B$.
% ===== END: sections/physical_world.tex =====
% ===== BEGIN: sections/building_blocks.tex =====
\section{Building blocks}
\label{sec:building-blocks}
\intuition{Before the dialogue begins, the agents share a vocabulary of planners and encoders that keeps proofs short and verifiable.}

\subsection*{Shortest paths via BFS}
Robot~A searches its current belief grid with breadth-first search. BFS expands the grid layer by layer, guaranteeing the shortest path when one exists. On a $10\times 10$ grid the search touches at most 100 cells and completes well under a millisecond.

\subsection*{Separators via potential-graph min-cut}
When no path emerges, Robot~A switches to a vertex-cut computation on the potential graph: each grid cell becomes a pair of nodes linked by capacity 1 if blocked or $\infty$ if free. A Dinic flow separates $s$ and $t$ by selecting the minimum set of blocked cells that disconnects the graph.

\subsection*{Path compression (RLE4)}
Instead of shipping raw coordinates, paths are encoded as direction runs. Two bits capture the heading (Up, Right, Down, Left) and six bits store the run length, packing up to 63 identical steps into a single byte. The sequence is then Base64 encoded for JSON transport.

\subsection*{Cut compression (DELTA16)}
Cut cells are stored as one absolute $(x,y)$ coordinate followed by signed int8 deltas relative to the previous cell. Large jumps trigger a restart marker that introduces another absolute coordinate. Typical separators of 6--12 cells use 12--20 bytes before Base64 encoding.

\subsection*{Witness bits}
Each separator cell carries one attribution bit: 0 means Robot~A detected the obstacle, 1 means Robot~B supplied it. Robot~B accepts the certificate if either it sees the obstacle locally or the witness bit credits Robot~A. The packed witness vector remains tiny (one bit per cell).

\subsection*{Checksums}
We use CRC16-CCITT in two places. The payload checksum (\texttt{digest16}) guards the raw run or cell bytes. The envelope checksum (\texttt{c}) runs over the canonical JSON string to catch transmission errors anywhere in the packet.
% ===== END: sections/building_blocks.tex =====
% ===== BEGIN: sections/running_example.tex =====
\section{Running example at a glance}
\label{sec:running-example}
\intuition{A single delivery attempt reveals the entire shape of the dialogue: a proposal, an optional correction, and a certificate that an oracle can replay.}

Figure~\ref{fig:maps} shows the partial views held by each robot during one of the benchmark seeds. Robot~A marks hot hazards along column three; Robot~B marks unstable pavement along column six. Neither robot knows that both hazard lines exist simultaneously.

\needspace{12\baselineskip}
\begin{figure}[htbp]
  \centering
  \begin{subfigure}{.32\linewidth}
    \centering
    \begin{verbatim}
A's map
R . . . . . . . . .
. H . . . . . . . .
. . . H . . . . . .
. . . H . . . . . .
. . . H . . . . . .
    \end{verbatim}
    \caption{Thermal hazards seen by Robot A}
  \end{subfigure}
  \begin{subfigure}{.32\linewidth}
    \centering
    \begin{verbatim}
B's map
R . . . . . . . . .
. . . . . . S . . .
. . . . . . S . . .
. . . . . . S . . .
. . . . . . S . . .
    \end{verbatim}
    \caption{Ground hazards seen by Robot B}
  \end{subfigure}
  \begin{subfigure}{.32\linewidth}
    \centering
    \begin{verbatim}
Union U
R . . . . . . . . .
. H . . . . S . . .
. . . H . . S . . .
. . . H . . S . . .
. . . H . . S . . .
    \end{verbatim}
    \caption{True world, known only to the oracle}
  \end{subfigure}
  \caption{Private and union views for a single $10\times 10$ layout. The restaurant is at $(0,0)$, the destination at $(9,9)$.}
  \label{fig:maps}
\end{figure}

\FloatBarrier

When the world is reachable, three messages suffice. Robot~A proposes a path encoded with RLE4. Robot~B validates the proposal against its own hazards, upgrades it to a certificate, and Robot~A sends an acknowledgement. Listing~\ref{lst:path-success} is the literal transcript for seed~247. Notice the indentation: the payload (`p`) contains the run string (`runs`) and its digest (`digest16`), while the envelope carries the sequence number (`q`) and the outer CRC (`c`).

\begin{lstlisting}[language=json,caption={Three-turn path success (seed 247).},label={lst:path-success}]
{
  "t": "PATH_PROPOSE",
  "p": {
    "runs": "QoVBgkKCRA==",
    "digest16": 22270
  },
  "s": "v1",
  "q": 0,
  "c": 16269
}
{
  "t": "PATH_CERT",
  "p": {
    "runs": "QoVBgkKCRA==",
    "digest16": 22270,
    "signed_by": ["B"]
  },
  "s": "v1",
  "q": 0,
  "c": 49459
}
{
  "t": "ACK",
  "p": {
    "ack_of": "PATH_CERT",
    "digest16": 22270
  },
  "s": "v1",
  "q": 1,
  "c": 3754
}
\end{lstlisting}

When the instance is impossible, the exchange still fits in three packets but the payload describes a separator. Listing~\ref{lst:cut-success} shows seed~123 where the cut cells come from both sensors. The witness field explains who attests each coordinate: the first two cells are supplied by Robot~A, the last two by Robot~B.

\begin{lstlisting}[language=json,caption={Three-turn cut certificate (seed 123).},label={lst:cut-success}]
{
  "t": "CUT_PROPOSE",
  "p": {
    "cells": "AAACAAEAAf4AAQ==",
    "k": 4,
    "digest16": 46585,
    "witness": "AA=="
  },
  "s": "v1",
  "q": 0,
  "c": 10141
}
{
  "t": "CUT_CERT",
  "p": {
    "cells": "AAACAAEAAf4AAQ==",
    "k": 4,
    "digest16": 46585,
    "witness": "AA==",
    "signed_by": ["B"]
  },
  "s": "v1",
  "q": 0,
  "c": 4584
}
{
  "t": "ACK",
  "p": {
    "ack_of": "CUT_CERT",
    "digest16": 46585
  },
  "s": "v1",
  "q": 1,
  "c": 36481
}
\end{lstlisting}

Occasionally the responder cannot certify a cut immediately. In those cases it replies with explicit conflicts and the initiator issues a single probe asking about the status of a handful of cells. The probe result updates the initiator's beliefs and the dialogue returns to the three-message pattern. Later sections explain how the belief updates and probe selection work.
% ===== END: sections/running_example.tex =====
% ===== BEGIN: sections/dialogue_flow.tex =====
\section{Dialogue flow}
\intuition{A conversation is a short alternation: propose a complete artifact, let the responder either certify it or reply with precise conflicts, and terminate on the acknowledgement of the certificate.}

\needspace{10\baselineskip}
\begin{figure}[htbp]
\centering
\begin{tabular}{@{}p{0.18\linewidth}p{0.64\linewidth}@{}}
\textbf{A} & \texttt{PATH\_PROPOSE} (84~B) \hfill \textbf{B} validates locally \\
\textbf{B} & \texttt{PATH\_CERT} (92~B) \hfill \textbf{A} receives signed proof \\
\textbf{A} & \texttt{ACK} (18~B) \hfill dialogue terminates \\
\end{tabular}

\medskip
\begin{tabular}{@{}p{0.18\linewidth}p{0.64\linewidth}@{}}
\textbf{A} & \texttt{PATH\_PROPOSE} (80~B) \\
\textbf{B} & \texttt{NACK} + conflicts (95~B) \\
\textbf{A} & \texttt{CUT\_PROPOSE} (118~B) \\
\textbf{B} & \texttt{CUT\_CERT} (126~B) \\
\textbf{A} & \texttt{ACK} (67~B) \\
\end{tabular}
\caption{Three-message fast path for reachable instances (top) and the five-message cut resolution for unreachable instances (bottom). Byte counts are medians from the SCHEMA-free 1k run.}
\label{fig:dialogue-flow}
\end{figure}
\FloatBarrier

\paragraph{Fast-path rule}
If the responder can fully validate a proposal, it replies with \texttt{PATH\_CERT} or \texttt{CUT\_CERT}. The initiator immediately acknowledges the certificate and the transcript ends; no trailing \texttt{DONE} messages are required.

\paragraph{Handling rejections}
On a \texttt{NACK}, the initiator updates its beliefs with the returned conflicts. If the current branch has not yet used its one probe, the initiator may send a \texttt{PROBE} with up to six ambiguous cells and folds the \texttt{PROBE\_REPLY} bits into its beliefs before recomputing a path or cut.

\paragraph{Message key map}
\begin{table}[h]
\centering
\begin{tabular}{ll|ll}
\toprule
Key & Meaning & Key & Meaning \\
\midrule
\texttt{t} & type & \texttt{p} & payload \\
\texttt{s} & schema version & \texttt{q} & sender sequence \\
\texttt{c} & CRC16 (envelope) & \texttt{d} & \texttt{digest16} payload CRC \\
\texttt{r} & path runs (RLE4) & \texttt{cs} & cut cells (DELTA16) \\
\texttt{k} & number of cut cells & \texttt{w} & witness bits (0=A, 1=B) \\
\texttt{sb} & signed\_by & \texttt{x} & conflicts in NACK payloads \\
\texttt{wht} & probe selector & & \\
\bottomrule
\end{tabular}
\caption{Compact keys used inside payloads. All examples in the paper adopt this mapping.}
\label{tab:key-map}
\end{table}
% ===== END: sections/dialogue_flow.tex =====
% ===== BEGIN: sections/communication_patterns.tex =====
\section{Communication patterns by system}
\intuition{Each system moves different information at different times. Seeing the flows side by side makes the byte and round differences intuitive.}

% We include PNGs generated from Mermaid (scripts/build_diagrams.sh).
% If images are not yet built, run the script or see ASCII in the repo under diagrams/.

\needspace{20\baselineskip}
\begin{figure}[htbp]
  \centering
  \begin{subfigure}{.49\linewidth}
    \centering
    \includegraphics[width=\linewidth]{figs/seq_certtalk.pdf}
    \caption{CertTalk: complete proposals, fast certificates, witness bits.}
  \end{subfigure}
  \hfill
  \begin{subfigure}{.49\linewidth}
    \centering
    \includegraphics[width=\linewidth]{figs/seq_sendall.pdf}
    \caption{Send-All: one-way raw streaming; union built at the responder.}
  \end{subfigure}
  \medskip
  \begin{subfigure}{.49\linewidth}
    \centering
    \includegraphics[width=\linewidth]{figs/seq_greedyprobe.pdf}
    \caption{Greedy-Probe: probes plus speculative proposals.}
  \end{subfigure}
  \hfill
  \begin{subfigure}{.49\linewidth}
    \centering
    \includegraphics[width=\linewidth]{figs/seq_respondermincut.pdf}
    \caption{Responder-MinCut: responder-led cut; one probe to reconcile.}
  \end{subfigure}
  \caption{Information flow across systems. Tags: [RAW] raw indices, [PROOF] path/cut artifact, [CONFLICTS] explicit blocking cells, [PROBE] targeted query.}
  \label{fig:patterns}
\end{figure}
\FloatBarrier
% ===== END: sections/communication_patterns.tex =====
% ===== BEGIN: sections/oracle.tex =====
\section{Oracle verification}
\intuition{The oracle is a referee that sees the union map. It replays path certificates to ensure every step is legal and replays cut certificates to ensure the barrier is genuine.}

For a path certificate the oracle decodes \texttt{runs}, verifies that the sequence starts at $s$ and ends at $t$, checks that each move is a Manhattan step within bounds, and rejects if any traversed cell belongs to $U$. Optional bookkeeping compares the certified path length to the oracle shortest path to compute the path-gap metric.

For a cut certificate the oracle decodes \texttt{cells}, confirms that the list length matches \texttt{k}, rebuilds witness bits, and rejects if any listed cell is outside the grid or free in $U$. It then removes the cut cells from the potential graph and checks that $s$ and $t$ fall in different components. Only then does the oracle accept the transcript. Because the oracle never interacts with the agents during the dialogue, archived certificates remain valid independently of the execution environment.
% ===== END: sections/oracle.tex =====
% ===== BEGIN: sections/agents.tex =====
\section{Agent behaviour}
\intuition{Both robots are finite-state planners. They track what the peer has asserted, keep a short memory of failed artifacts, and decide between paths and cuts based on that history.}

Robot~A begins each conversation with an empty belief map and a halo that discourages retrying recently blocked cells. After every message it integrates the peer's evidence. Belief updates are monotone: once a cell is believed blocked it stays blocked, and once it is believed free it stays free. Failed digests are stored so that identical proposals are never retried.

When it is Robot~A's turn to speak, it first constructs a candidate grid by starting from its private mask, adding all cells in the blocked belief set, removing all cells in the free belief set, and temporarily marking the halo cells as blocked. A breadth-first search over this grid either discovers a viable path or proves that no such path exists within the configured horizon. If a path exists, Robot~A serializes it with RLE4 and emits \texttt{PATH\_PROPOSE}. If no path survives, Robot~A computes a minimum separator on the potential graph using Dinic's algorithm with vertex splitting, encodes the separator with DELTA16, adds the witness bits, and emits \texttt{CUT\_PROPOSE}. When a \texttt{NACK} arrives, the conflicts enlarge the belief sets and may trigger a single probe before another artifact is attempted.

Robot~B has a symmetric planner. Upon receiving \texttt{PATH\_PROPOSE} it decodes the run string, reconstructs the path, and checks every step against its map and belief sets. A successful path is immediately returned as \texttt{PATH\_CERT}; a failure yields \texttt{NACK} with the offending coordinates. Upon receiving \texttt{CUT\_PROPOSE}, Robot~B decodes the separator, validates witness bits, verifies local consistency, and ensures that removing the separator from its potential graph disconnects $s$ from $t$. A successful separator is returned as \texttt{CUT\_CERT}. If some cells cannot be justified, Robot~B responds with \texttt{NACK} and highlights those cells so Robot~A can probe or recompute. When Robot~B receives a probe it answers with a packed bitmap describing which queried cells are blocked in its map.

Both robots terminate once they have sent or acknowledged a certificate. The conversation history at that point consists of a handful of JSON objects that capture the entire argument. The next section explains how the oracle replays those objects to provide an independent verdict.
% ===== END: sections/agents.tex =====
% ===== BEGIN: sections/encodings.tex =====
\section{Encodings and envelope}
\label{sec:encodings}
\intuition{Keeping packets tiny means reusing one envelope schema and compact payload encodings for both artefacts and the bookkeeping messages that surround them.}

Every message has the canonical form
\[
  m = \{\texttt{t}:\tau,\;\texttt{p}:\pi,\;\texttt{s}:"v1",\;\texttt{q}:q,\;\texttt{c}:c\}.
\]
The type $\tau$ guards the control flow, \texttt{p} wraps either an artifact or a housekeeping payload, \texttt{q} is a monotone sequence counter per sender, and \texttt{c} is the CRC16-CCITT of the canonical JSON rendering. The same envelope is used for proposals, certificates, NACKs, probes, and acknowledgements.

\subsection*{Paths}
\begin{itemize}
\item \texttt{runs} stores the RLE4 direction stream described in \autoref{sec:building-blocks}. A sequence such as Right×3, Down×4, Right×2, Down×1 fits into four bytes before Base64.
\item \texttt{digest16} is the CRC16 of the raw run bytes. The responder recomputes the checksum before decoding to catch corruption.
\item \texttt{signed\_by} lists which robot issued the certificate. For proposals the field is absent.
\end{itemize}

\subsection*{Cuts}
\begin{itemize}
\item \texttt{cells} carries the DELTA16 encoding of the separator. The initial absolute coordinate plus the delta stream usually compress ten cells into 18--20~bytes.
\item \texttt{k} is a consistency check for the decoder.
\item \texttt{witness} packs one bit per cell. The responder concedes peer-only evidence when the witness credits that peer.
\item The same \texttt{digest16} and optional \texttt{signed\_by} fields play the roles they did for paths.
\end{itemize}

\subsection*{Supporting messages}
\begin{itemize}
\item \texttt{ACK} echoes the digest of the artifact it acknowledges.
\item \texttt{NACK} carries either a bit-mask (for path conflicts) or an explicit list of conflicting coordinates (for cuts).
\item \texttt{PROBE} lists up to six ambiguous cells under the \texttt{wht} (“what”) selector and \texttt{PROBE\_REPLY} returns the blocked/free bits in the same order. Only one probe is allowed per branch.
\end{itemize}
% ===== END: sections/encodings.tex =====
% ===== BEGIN: sections/system_model.tex =====
\section{Problem setting}
\intuition{Two finite-state agents alternate messages. Each sees only part of the obstacle map. Their job is to settle the reachability question without exceeding the byte budget.}

Cells are indexed by integer pairs $(x,y)$ with $x$ increasing eastward and $y$ increasing southward. Movement is four-neighbour only; diagonal steps never occur.

We model the environment as a directed grid graph with vertex set $V = \{0,\ldots,9\}^2$ and edges that connect four-neighbour cells. Robot~A knows a private mask $A$ marking thermal hazards. Robot~B knows a private mask $B$ marking structural hazards. The world that matters is the union $U = A \cup B$, yet neither robot ever observes $U$ directly.

The start cell $s$ and goal cell $t$ are common knowledge. The communication channel is perfectly reliable but stingy: each conversation must finish within sixty-four messages, no individual packet may exceed 256 bytes, and the transcript as a whole must stay under 3{,}072 bytes. Messages alternate between robots; Robot~A initiates every run. Success is defined semantically rather than syntactically: a transcript is successful only if it terminates within budget and the final artifact passes the oracle's verification on $U$.

Both robots are deterministic. Their internal state consists of the private map, the growing catalogue of what the peer has claimed, and a small halo that nudges replanning away from recently blocked cells. There is no randomness or machine learning involved. The oracle is an offline checker that never intervenes during the dialogue. It simply replays the final artifact against the union grid and declares acceptance or rejection.
% ===== END: sections/system_model.tex =====
% ===== BEGIN: sections/experimental_setup.tex =====
\section{Experimental setup}
\intuition{A fixed cache of 1{,}000 layouts lets us run every system side by side. All code is deterministic; every number in the paper traces back to one of the JSONL files.}

\textbf{Dataset.} We reuse the publicly released cache \texttt{data/20251008T151417Z\_cache.jsonl}. Each entry records the private masks $A$ and $B$, the oracle reachable flag for $U$, the optimal path length when reachable, and the minimum cut size otherwise. Generating the cache is a single command and takes under a minute on a laptop. The cache seed is fixed so future runs remain comparable.

\textbf{Systems compared.} We evaluate four deterministic policies under identical limits: CertTalk (this work), the baseline Send-All that streams blocked indices in $32$-cell chunks, Greedy-Probe which mirrors the early path-vs-probe heuristic, and Responder-MinCut where Agent~B leads with a cut-first strategy. All systems use the compact schema toggle and the same conversation budgets.

\textbf{Metrics.} For each transcript we log success (oracle accepted certificate), bytes, rounds, path gap, cut gap, and the share of conversations ending in a certificate (equal to success for deterministic systems). Medians and 95\% bootstrap intervals are reported across the full cache. Because every transcript is archived, the statistics are easy to audit: recomputing the median bytes is a matter of streaming the JSONL file.
% ===== END: sections/experimental_setup.tex =====
% ===== BEGIN: sections/results.tex =====
\section{Results}
\intuition{After removing the SCHEMA handshake and enabling the compact key map everywhere, CertTalk matches the deterministic baselines on success while closing most of the byte gap with Send-All.}

\begin{table}[h]
\centering
\caption{Summary across 1{,}000 cached instances (SCHEMA-free run on 2025-10-09).}
\label{tab:medians}
\begin{tabular}{lcccc}
\toprule
System & Success & Median bytes & Median rounds & Interpretability \\
\midrule
CertTalk & 1.000 & 468.5 & 5.0 & 1.000 \\
Send-All & 1.000 & 364.0 & 3.0 & 1.000 \\
Greedy-Probe & 1.000 & 468.5 & 5.0 & 1.000 \\
Responder-MinCut & 0.205 & 506.0 & 6.0 & 0.205 \\
\bottomrule
\end{tabular}
\end{table}

CertTalk now attains the same 100\% success rate as Send-All and Greedy-Probe with a median transcript size of $468.5$~bytes. The three-turn fast path fires on 23.5\% of instances, and 99.5\% of runs finish within seven messages. Responder-MinCut remains intentionally conservative: it only succeeds on unreachable instances, providing a deterministic cut-only baseline for qualitative comparison.

The transcript gallery in \autoref{app:gallery} illustrates the typical behaviours: a three-message path certificate, a three-message cut certificate, and a five-message probe branch before certification. The forthcoming figures will add full byte and round distributions; the tabulation above already spells out the key medians the paper highlights.

\needspace{12\baselineskip}
\begin{figure}[htbp]
  \centering
  \includegraphics[width=\linewidth]{figs/bytes_by_system.pdf}
  \caption{Communication cost. Total bytes per transcript across four systems. Boxes show the distribution; faint points are individual runs.}
  \label{fig:bytes}
\end{figure}

\needspace{12\baselineskip}
\begin{figure}[htbp]
  \centering
  \includegraphics[width=\linewidth]{figs/rounds_by_system.pdf}
  \caption{Dialogue length. Number of messages to termination. CertTalk’s fast certificate reply collapses many runs to three messages.}
  \label{fig:rounds}
\end{figure}

\needspace{12\baselineskip}
\begin{figure}[htbp]
  \centering
  \includegraphics[width=\linewidth]{figs/byte_mix_stacked.pdf}
  \caption{What the bytes buy. Share of bytes by message type. Send-All spends nearly all bytes on raw blocks; CertTalk spends on complete proof artifacts.}
  \label{fig:mix}
\end{figure}
\FloatBarrier
% ===== END: sections/results.tex =====
% ===== BEGIN: sections/ablations.tex =====
\section{Ablations and evolution}
\intuition{Each protocol tweak was validated on the same 1{,}000-instance cache, making it easy to attribute byte and round improvements to individual ideas.}

\paragraph{Witness bits \& targeted probes}
The 2025-10-08 run without witness bits stalled at a median of $3.1$~KB and $21$ rounds. Introducing witness bits and the single-probe rule dropped CertTalk and Greedy-Probe to $947$~bytes and $7$ rounds while preserving $>99.9\%$ success. Conflicting cells now resolve on the very next attempt rather than oscillating.

\paragraph{SCHEMA handshake removal}
Eliminating the two-message SCHEMA preamble pushed the median down to $653.5$~bytes and kept rounds at $7$. No correctness regressions were observed; the same single probe still sufficed on the ambiguous cuts.

\paragraph{Compact schema everywhere}
Applying the compact key map to all systems closed the final gap. CertTalk’s median bytes now sit at $468.5$ with a median of 5~rounds, Greedy-Probe mirrors the same numbers, and Send-All benefits as well (median $364$~bytes). The responder-led min-cut baseline remains unchanged because it does not rely on compact keys.

Future drafts will add a small bar chart summarising the byte reductions from each toggle. The text above already records the concrete medians for reproducibility.
% ===== END: sections/ablations.tex =====
% ===== BEGIN: sections/limitations.tex =====
\section{Limitations}
\intuition{The protocol deliberately targets a narrow, fully deterministic setting; several assumptions would need adjustment for broader deployments.}

\begin{itemize}
\item \textbf{Reliable channel.} Messages never drop or reorder in our experiments. A lossy link would require replay buffers or per-message signatures.
\item \textbf{Two agents only.} CertTalk currently handles a single initiator and responder. Extending to $n$ agents would require a richer notion of witness attribution and shared sequence numbers.
\item \textbf{Small grids.} The $10\times10$ benchmark keeps planning trivial so that communication dominates. Larger maps are likely to need chunked certificates or multi-stage artifacts.
\item \textbf{No privacy model.} Witness bits expose which robot attested each obstacle. Protecting sensor privacy would require masking or cryptographic commitments.
\end{itemize}
% ===== END: sections/limitations.tex =====
% ===== BEGIN: sections/reproducibility.tex =====
\section{Reproducibility}
\intuition{All artifacts live in the repository; reproducing the study is a sequence of four commands once \texttt{uv} is installed.}

\begin{enumerate}
\item Create the environment and install the package:
\begin{lstlisting}[language=bash]
uv venv
source .venv/bin/activate
uv pip install -e .[analysis]
\end{lstlisting}

\item Regenerate the 1{,}000-instance cache:
\begin{lstlisting}[language=bash]
uv run python -m agent_talk.env.generator \
  --out data/20251008T151417Z_cache.jsonl \
  --n 1000 --size 10 --seed 123
\end{lstlisting}

\item Re-run all four systems under the SCHEMA-free configuration:
\begin{lstlisting}[language=bash]
uv run python -m agent_talk.runners.batch_eval --cache data/20251008T151417Z_cache.jsonl --system certtalk --out runs/20251009T201009Z_certtalk.jsonl
uv run python -m agent_talk.runners.batch_eval --cache data/20251008T151417Z_cache.jsonl --system sendall --out runs/20251009T201024Z_sendall.jsonl
uv run python -m agent_talk.runners.batch_eval --cache data/20251008T151417Z_cache.jsonl --system greedyprobe --out runs/20251009T201034Z_greedyprobe.jsonl
uv run python -m agent_talk.runners.batch_eval --cache data/20251008T151417Z_cache.jsonl --system respondermincut --out runs/20251009T201043Z_respondmincut.jsonl
\end{lstlisting}

\item Aggregate metrics into the summary table:
\begin{lstlisting}[language=bash]
uv run python -m agent_talk.analysis.metrics \
  --inputs runs/20251009T201009Z_certtalk.jsonl \
           runs/20251009T201024Z_sendall.jsonl \
           runs/20251009T201034Z_greedyprobe.jsonl \
           runs/20251009T201043Z_respondmincut.jsonl \
  --out runs/20251009T201059Z_summary.csv
\end{lstlisting}
\end{enumerate}

The transcript gallery in \autoref{app:gallery} references the exact seeds used in the figures; every JSONL file lives under \texttt{runs/} for independent inspection.
% ===== END: sections/reproducibility.tex =====
% ===== BEGIN: sections/related_work.tex =====
\section{Related work}
\begin{intuition}
Our protocol sits at the intersection of distributed verification, multi-agent pathfinding, and interpretable communication. We borrow the idea of proof-carrying artifacts from certifying algorithms and adapt it to a bandwidth-constrained, two-agent dialog.
\end{intuition}

\paragraph{Certifying algorithms.} Classical certifying algorithms output a solution accompanied by a proof object that can be checked in linear time. Typical examples include spanning tree certification, max-flow/min-cut verifiers, and witness objects for NP-complete problems. CertTalk adopts the same philosophy: the final packet is sufficient to convince an independent oracle of reachability or separation.

\paragraph{Multi-agent pathfinding.} Distributed approaches to multi-agent pathfinding generally fall into two categories: global planners that share full maps or policies, and local planners that communicate heuristics or priorities. Our contribution shows that a third point in the design space exists: agents can trade minimally sufficient proofs instead of raw maps or unstructured hints, achieving comparable reliability with drastically smaller payloads.

\paragraph{Emergent communication.} Learned protocols often produce opaque signals that are difficult to audit. In contrast, CertTalk intentionally limits itself to typed JSON messages with human-readable semantics. This constraint is motivated by safety review: the logs can be inspected line by line to understand exactly why a decision was made.
% ===== END: sections/related_work.tex =====
% ===== BEGIN: sections/conclusion.tex =====
\section{Conclusion}
\begin{intuition}
A handful of typed messages can replace full map exchange if the messages themselves carry proofs. CertTalk demonstrates the idea in a controlled setting, yielding transcripts that humans and machines can audit alike.
\end{intuition}

We presented CertTalk, a certificate-carrying protocol for two agents with private occupancy grids. Witness bits and fast certificate replies collapse dialogue length, while compact encodings keep every packet well under $256$ bytes. On a reproducible 1{,}000-instance benchmark CertTalk matches the reliability of the Send-All baseline and stays within $105$ bytes of its median payload, all while producing logs that are literal, interpretable JSON objects. We hope this work encourages the broader community to treat communication products as proofs and to demand the same auditability from robotic dialogues as we do from certifying algorithms.
% ===== END: sections/conclusion.tex =====
\appendix
% ===== BEGIN: sections/appendix_gallery.tex =====
\section{Transcript gallery}
\label{app:gallery}
\begin{table}[h]
  \centering
  \caption{Representative transcripts with compact payloads. Each message is copied verbatim from the 2025-10-09T20:10Z run.}
  \label{tab:gallery}
  \begin{tabular}{p{0.18\linewidth}p{0.77\linewidth}}
    \toprule
    Scenario & Transcript \\
    \midrule
    Reachable (seed 247) & \begin{minipage}[t]{\linewidth}\begin{lstlisting}[language=json]
{"c":16269,"p":{"d":22270,"r":"QoVBgkKCRA=="},"q":0,"s":"v1","t":"PATH_PROPOSE"}
{"c":49459,"p":{"d":22270,"r":"QoVBgkKCRA==","sb":["B"]},"q":0,"s":"v1","t":"PATH_CERT"}
{"c":3754,"p":{"a":"PATH_CERT","d":22270},"q":1,"s":"v1","t":"ACK"}
\end{lstlisting}\end{minipage} \\
    \midrule
    Unreachable (seed 123) & \begin{minipage}[t]{\linewidth}\begin{lstlisting}[language=json]
{"c":10141,"p":{"cs":"AAACAAEAAf4AAQ==","d":46585,"k":4,"w":"AA=="},"q":0,"s":"v1","t":"CUT_PROPOSE"}
{"c":4584,"p":{"cs":"AAACAAEAAf4AAQ==","d":46585,"k":4,"sb":["B"]},"q":0,"s":"v1","t":"CUT_CERT"}
{"c":36481,"p":{"a":"CUT_CERT","d":46585},"q":1,"s":"v1","t":"ACK"}
\end{lstlisting}\end{minipage} \\
    \midrule
    Probe before certificate (seed 624) & \begin{minipage}[t]{\linewidth}\begin{lstlisting}[language=json]
{"c":31020,"p":{"d":56648,"r":"QYFEgUOFQYI"},"q":0,"s":"v1","t":"PATH_PROPOSE"}
{"c":14719,"p":{"mask":"FAUB","n":"PATH_PROPOSE","reason":"BLOCKED"},"q":0,"s":"v1","t":"NACK"}
{"c":62015,"p":{"cs":"BwAJAAH/Af8=","d":10962,"k":3,"w":"Bg=="},"q":1,"s":"v1","t":"CUT_PROPOSE"}
{"c":41997,"p":{"cs":"BwAJAAH/Af8=","d":10962,"k":3,"sb":["B"]},"q":1,"s":"v1","t":"CUT_CERT"}
{"c":52641,"p":{"a":"CUT_CERT","d":10962},"q":2,"s":"v1","t":"ACK"}
\end{lstlisting}\end{minipage} \\
    \bottomrule
  \end{tabular}
\end{table}
% ===== END: sections/appendix_gallery.tex =====

\bibliographystyle{unsrtnat}
\bibliography{references}

\end{document}
